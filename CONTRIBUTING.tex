\documentclass[a4paper, 10pt]{article}
\pagestyle{plain}

% Set the space between text and footnotes
\addtolength{\skip\footins}{2pc plus 5pt}

% Enable simultaneous Greek and English writing
\usepackage[utf8]{inputenc}
\usepackage[greek, english]{babel}
\usepackage{alphabeta}

% My fixed-width fonts
% \usepackage[T1]{fontenc}
% \usepackage{inconsolata}

\usepackage{graphicx}
\usepackage{fullpage}
\usepackage{amsmath}
\usepackage{amsthm}
\usepackage{amssymb}
\usepackage{xspace}

\usepackage[dvipsnames]{xcolor}
\usepackage{listings}
\usepackage{amssymb}

\usepackage{caption}
\usepackage{subcaption}

\newtheorem{proposition}{Proposition}[section]
\newtheorem{lemma}[proposition]{Lemma}


\linespread{1.2}
\setlength{\parindent}{0pt}
\setlength{\parskip}{12pt}

\definecolor{mygray}{gray}{0.5}
\definecolor{myblue}{rgb}{.1 .5 .6}

\usepackage{hyperref}

\hypersetup{
    colorlinks = true,
    linkcolor = mygray,
    filecolor = myblue,
    urlcolor = myblue
}

% The order of the package import DOES count. If geometry is used at the
% beginning, margins are not set
\usepackage[a4paper, margin=1.2in]{geometry}

% \newfontfamily\custommono[LetterSpace=5]{Courier New}

% Code fragments inside text
\usepackage{listings}
\lstdefinestyle{all}{
    % numbers=left,
    basicstyle=\ttfamily\small,
    breaklines=true,
    numbersep=4pt,
    tabsize=4,
    showstringspaces=false,
    captionpos=b
}

\begin{document}
\title{Reading and Contribution Notes}
\author{Christos Malliopoulos}
\maketitle
\tableofcontents



\section{Repository forking and cloning}
I forked \texttt{supermassive-intelligence/scalarlm} (public repo) to \texttt{chris-relational/smi-scalarlm} (this was done from the \texttt{supermassive-intelligence/scalarlm} page on GitHub).

I cloned \texttt{supermassive-intelligence/scalarlm} locally under \texttt{remotes/chris-relational} as \texttt{smi-scalarlm} (that is, I used the name of the original repository).

I did all these to be able to fetch from the original repo and push to the forked repo:  
\begin{lstlisting}[style=all]
$ cd smi-scalarlm
$ git remote set-url origin --push git@chris-relational-github:chris-relational/smi-scalarlm.git
\end{lstlisting}

Now I have a local repository with an upstream that fetches from `massive-intelligence/scalarlm` and pushes to 
`chris-relational/smi-scalarlm`.



...

\end{document}
